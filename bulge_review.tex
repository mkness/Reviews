% style notes
% - \,percent not \%

\documentclass[12pt, preprint]{aastex}
\usepackage{bm, graphicx, subfigure, amsmath, morefloats}

% words
\newcommand{\project}[1]{\textsl{#1}}
\newcommand{\thecannon}{\project{The~Cannon}} 
\newcommand{\tc}{\project{The~Cannon}} 
\newcommand{\apogee}{\project{APOGEE}}
\newcommand{\apokasc}{\project{APOKASC}}
\newcommand{\aspcap}{\project{ASPCAP}}
\newcommand{\corot}{\project{Corot}}
\newcommand{\kepler}{\project{Kepler}}
\newcommand{\gaia}{\project{Gaia}}
\newcommand{\gaiaeso}{\project{Gaia--ESO}}
\newcommand{\galah}{\project{GALAH}}
\newcommand{\most}{\project{MOST}}
\newcommand{\code}[1]{\texttt{#1}}
\newcommand{\documentname}{\textsl{Article}}

\newcommand{\teff}{\mbox{$\rm T_{eff}$}}
\newcommand{\kms}{\mbox{$\rm kms^{-1}$}}
\newcommand{\feh}{\mbox{$\rm [Fe/H]$}}
\newcommand{\xfe}{\mbox{$\rm [X/Fe]$}}
\newcommand{\alphafe}{\mbox{$\rm [\alpha/Fe]$}}
\newcommand{\mh}{\mbox{$\rm [M/H]$}}
\newcommand{\logg}{\mbox{$\rm \log g$}}
\newcommand{\noise}{\sigma_{n\lambda}}
\newcommand{\scatter}{s_{\lambda}}
\newcommand{\pix}{\mathrm{pix}}
\newcommand{\rfn}{\mathrm{ref}}
\newcommand{\rgc}{\mbox{$\rm R_{GC}$}}
\newcommand{\rgal}{\mbox{$\rm R_{GAL}$}}
\newcommand{\vgal}{\mbox{$\rm V_{GAL}$}}

% math and symbol macros
\newcommand{\set}[1]{\bm{#1}}
\newcommand{\starlabel}{\ell}
\newcommand{\starlabelvec}{\set{\starlabel}}
\newcommand{\mean}[1]{\overline{#1}}
\newcommand{\given}{\,|\,}
%\newcommand{\teff}{\mbox{$\rm T_{eff}$}}
%\newcommand{\kms}{\mbox{$\rm kms^{-1}$}}
%\newcommand{\feh}{\mbox{$\rm [Fe/H]$}}
%\newcommand{\xfe}{\mbox{$\rm [X/Fe]$}}
%\newcommand{\alphafe}{\mbox{$\rm [\alpha/Fe]$}}
%\newcommand{\mh}{\mbox{$\rm [M/H]$}}
%\newcommand{\logg}{\mbox{$\rm \log g$}}
%\newcommand{\noise}{\sigma_{n\lambda}}
%\newcommand{\scatter}{s_{\lambda}}

% math
\newcommand{\numax}{$\nu_{\max}$}
\newcommand{\deltanu}{$\Delta\nu$}
\bibliographystyle{apj}

\begin{document}



% To do. 
% 1. release two catalogues :
% 1a) trained on log mass and 1b) trained on log mass
%2. resample the errors in label space - add a random error to the mass term and rerun the labels through the cannon and see if the standard deviation and bias in the take one out test is any different - if not we know the noisiness in the training set is not an issue
%3. cut out a lot of regions except for the CN and retrain and see if the take-out-out test sigma increases or decreases in the dispersion - if increases we know we are limited by the information available in the data. If decreases there is degeneracy / noisiness in the covariance of labels? 

% plotting in /Apogee_ages/makeplot_scatter_test18_step.py 
% and plotelements_on_spectra.py 

%\title{Measuring red-giant masses and ages from stellar spectra: Stellar Ages across the Milky Way disk}
\title{Masses and ages from stellar spectra -- a catalogue of APOGEE red clump and red giant ages for the Milky Way disk}
\author{M.~Ness\altaffilmark{1},
David~W.~Hogg\altaffilmark{1,2,3},
H.-W.~Rix\altaffilmark{1},
M.~Martig \altaffilmark{1},
A.Y.~Q~Ho \altaffilmark{1},
\textbf{others}}
\altaffiltext{1}{Max-Planck-Institut f\"ur Astronomie, K\"onigstuhl 17, D-69117 Heidelberg, Germany}
\altaffiltext{2}{Center for Cosmology and Particle Physics, Department of Phyics,
             New York University, 4 Washington Pl., room 424, New York, NY, 10003, USA}
\altaffiltext{3}{Center for Data Science, New York University, 726 Broadway, 7th Floor, New York, NY 10003, USA}
% \altaffiltext{4}{NSF Astronomy and Astrophysics Postdoctoral Fellow}
% \altaffiltext{5}{Department of Physics \& Astronomy, Johns Hopkins University, Baltimore, MD, 21218, USA}
\email{ness@mpia.de}

Metallicity distribution of bulge stars . . . . . . . . . . . . . . Melissa Ness & Ken Freeman
1. Global MDF: - Discussion on the overall metallicity, gradients
2. Formation models and the MDF: - Discussion of how models relate to observable
trends in the MDF
3. Multiple bulge populations: - Discussion on multiple populations with different
groups associating them to different origins
4. The MFD in the plane: - i.e. new results from APOGEE that are in hand from
analysis of public data.
5. Disentangling populations with kinematics, individual abundances and proper
motions: - how the additional phase-space parameters can be used to differentiate
formation histories
6. The metal poor [Fe/H] � -1.0 population in the bulge: - Discussion about the
RR Lyrae distribution and also what the metal poor stars in the inner region may be: distinct bulge population versus halo population.

\begin{abstract}%

\end{abstract}

\section*{Acknowledgments}
It is a pleasure to thank ABandC.


\bibliography{tc.bib}

\end{document}
